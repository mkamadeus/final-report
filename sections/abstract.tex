\clearpage
\phantomsection{}
\addcontentsline{toc}{chapter}{Abstrak}
\begin{center}
  \textbf{\large \MakeUppercase{Abstrak}}\\[3em]
\end{center}

% latar belakang
Dalam pengembangan sistem pemelajaran mesin, proses dilakukan secara iteratif untuk mengembangkan model secara bertahap.
Semakin cepat proses tersebut dilakukan, semakin banyak iterasi yang bisa dilakukan untuk mengembangkan sistem dengan model yang lebih mutakhir.

% hasil penilitian
Oleh karena itu, diperlukan suatu cara untuk mempercepat proses tersebut.
Salah satu pendekatan untuk mempercepat proses tersebut ada pada tahapan pengembangan sistem pemelajaran mesin.
Penulis merancang sebuah kakas yang dapat mengotomatisasi proses tersebut melalui pembuatan kakas yang menghasilkan kode yang siap dijalankan, sehingga sistem dapat diubah secara fleksibel.

% metode penilitian
Pengembangan kakas ini dilakukan setelah melakukan analisis terhadap hal-hal yang perlu dilakukan dalam proses pengembangan sistem.
Pengujian kakas ini dilakukan melalui studi kasus eksperimen pemelajaran mesin yang beragam, baik dari format masukan hingga jenis permasalahannya.

% kesimpulan
Kakas ini memiliki potensi untuk dikembangkan lebih lanjut.
Peninjauan lanjut diperlukan agar kakas ini dapat bekerja dalam kasus-kasus lainnya yang belum dibahas dalam makalah ini.

\noindent \textbf{Kata kunci:}\newline
\emph{MLOps, sistem pemelajaran mesin, konversi eksperimen}
