\clearpage
\phantomsection{}
\addcontentsline{toc}{chapter}{Abstrak}
\begin{center}
  \textbf{\large \MakeUppercase{Abstrak}}\\[3em]
\end{center}

% latar belakang
Dalam pengembangan suatu sistem pemelajaran mesin, pengembangan biasanya dilakukan secara iteratif untuk mengembangkan model.
Semakin cepat suatu proses dalam sebuah tahapan dilakukan, semakin banyak iterasi yang bisa dilakukan untuk mengembangkan sistem dengan model yang lebih mutakhir pada setiap iterasinya.
Dalam setiap iterasi, terdapat beberapa hal yang dapat menghambat atau memperlambat proses tersebut.

% hasil penilitian
Oleh karena itu, diperlukan suatu cara untuk mempercepat iterasi pengembangan sistem.
Salah satu pendekatan membantu dalam tahapan pengembangan sistem pemelajaran mesin.
Penulis merancang sebuah kakas yang dapat mengotomatisasi proses tersebut melalui pembuatan kakas yang menghasilkan kode yang siap dijalankan, sehingga sistem dapat diubah secara fleksibel.
Kakas ini disebut sebagai \monospace{myx}, yang merupakan kakas konversi dari \textit{file-file} eksperimen menjadi sistem yang siap dijalankan berbasiskan \textit{code generation}.

% metode penilitian
Pengembangan kakas ini dilakukan setelah melakukan analisis terhadap hal-hal yang perlu dilakukan dalam proses pengembangan sistem.
Pengujian kakas ini dilakukan melalui studi kasus eksperimen pemelajaran mesin yang beragam, baik dari format masukan hingga jenis permasalahannya.
Adapun studi kasis yang diuji adalah untuk data berjenis tabular dan data berjenis citra pada kombinasi model pemelajaran mesin yang berbeda.

\noindent \textbf{Kata kunci:}\newline
\emph{MLOps, sistem pemelajaran mesin, konversi eksperimen}
