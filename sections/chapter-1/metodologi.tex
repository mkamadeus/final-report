\section{Metodologi}

Berikut adalah tahapan atau metodologi yang dilakukan dalam proses penulisan makalah ini.

\begin{enumerate}
  \item Melakukan penentuan masalah secara umum
  
  Pada tahapan awal, masalah terkait proses konversi dari eksperimen ke sistem akan ditentukan.
  Masalah yang ditemukan akan dipilah berdasarkan hal yang paling berpengaruh terhadap proses tersebut untuk akan dicari solusinya.

  \item Melakukan studi literatur dan eksplorasi kakas
  Penulis akan melakukan pencarian informasi terkait metode-metode \textit{model serving} dalam sistem pemelajaran mesin dari masalah yang ada dalam pembuatan sistem.
  Melalui tahapan ini, berbagai metode otomatisasi akan dibandingkan untuk mendapatkan solusi yang optimal.
  Dalam tahapan ini juga akan dilakukan beberapa studi untuk literatur ataupun kakas terkait yang menunjang metode otomatisasi dalam makalah ini.

  \item Melakukan pembuatan rancangan solusi
  
  Berdasarkan permasalahan yang ada, sebuah rancangan solusi yang dapat menyelesaikan beberapa poin masalah akan diajukan.
  Rancangan solusi dapat didasarkan dari sumber bacaan yang ditemui penulis atau dari ide penulis.
  Rancangan juga dapat berupa perkembangan dari suatu literatur atau kakas yang ditemui penulis.

  \item Melakukan implementasi dari proposal solusi yang telah dirancang
  
  Rancangan solusi yang sudah diajukan akan diimplementasikan.
  Selain melakukan implementasi, dalam tahapan ini juga akan dilakukan pengembangan beberapa alternatif solusi secara teknis.
  Hal tersebut bertujuan untuk mencari implementasi yang baik dari aspek pengembangan dan dalam penyelesaian solusinya.
  Implementasi akan dilakukan secara iteratif bergantung pada hasil pengujian.

  \item Melakukan pengujian terhadap implementasi proposal solusi terhadap contoh yang ada di dunia nyata
  
  Berhubungan dengan tahapan sebelumnya, rancangan solusi yang diimplementasikan akan diuji pada beberapa studi kasus.
  Studi kasus yang diambil merupakan kasus yang umum ditemui dalam pembangunan sistem pemelajaran mesin.
  
\end{enumerate}