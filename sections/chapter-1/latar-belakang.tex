\section{Latar Belakang}

Dalam proses pengembangan sistem dengan teknologi kecerdasan buatan, biasanya akan melewati tahapan yang cukup panjang.
Dari proses pengumpulan data hingga proses pembuatan sistem inferensi, hal-hal yang dilakukan tersebut cenderung mengikuti suatu alur pekerjaan atau \textit{workflow} tertentu.
Dalam proses pembuatannya, biasanya terdapat beberapa proses yang memakan waktu, salah satunya adalah dalam proses konversi sistem inferensi dari eksperimen-eksperimen yang dibuat.

Oleh karena itu, dalam laporan ini, penulis akan membahas terkait hal-hal apa saja yang bisa diterapkan dalam proses pengembangan \textit{pipeline machine learning} sehingga proses tersebut bisa lebih efisien dan cepat.