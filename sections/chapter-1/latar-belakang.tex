\section{Latar Belakang}

Dalam proses pengembangan sistem pemelajaran mesin, biasanya banyak tahapan yang perlu dilewati untuk memgembangkan sistem.
Dari proses pengumpulan data hingga proses \textit{delivery} dari sistem inferensi dilakukan berdasarkan kaidah-kaidah yang ada dalam MLOps.
Terdapat beberapa tahapan yang memakan waktu dalam proses pembuatan sistem inferensi, salah satunya adalah bagaimana sistem tersebut dibangun dari eksperimen-eksperimen yang telah dilakukan sebelumnya.

Salah satu yang menjadi perhatian penulis adalah proses konversi dari eksperimen menjadi sistem yang utuh.
Dalam proses pembuatan sistem, banyak yang perlu diperhatikan dari sisi pengembangan perangkat lunak.
Hal-hal tersebut tentunya di luar dari cakupan eksperimen model dalam membuat layanan dengan teknologi pemelajaran mesin.

Dalam makalah ini, penulis akan membahas hal apa saja yang bisa diterapkan dalam proses pengembangan sistem pemelajaran mesin sehingga proses tersebut dapat dikerjakan dengan lebih efisien.
Khususnya, makalah ini akan membahas teknologi yang bisa dikembangkan untuk melakukan konversi dari eksperimen-eksperimen yang dibuat menjadi satu sistem yang koheren dan \textit{production-ready}.