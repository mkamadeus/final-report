\section{Latar Belakang}

Dalam proses pengembangan sistem pembelajaran mesin, biasanya banyak tahapan yang perlu dilewati untuk memgembangkan sistem.
Dari proses pengumpulan data hingga proses \textit{delivery} dari sistem inferensi, hal-hal yang dilakukan berdasarkan kaidah-kaidah yang ada dalam MLOps.
Terdapat beberapa proses yang memakan waktu dalam proses pembuatannya, salah satunya adalah bagaimana sistem tersebut dibangun dari eksperimen-eksperimen yang dilakukan.

Dalam makalah ini, penulis akan membahas terkait hal-hal apa saja yang bisa diterapkan dalam proses pengembangan sistem pembelajaran mesin sehingga proses tersebut dapat dikerjakan dengan lebih efisien.
Khususnya, akan dibahas terkait teknologi yang bisa dikembangkan untuk melakukan konversi dari eksperimen-eksperimen yang dibuat menjadi satu sistem yang koheren dan \textit{production-ready}.