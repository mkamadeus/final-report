\section{Latar Belakang}

Dalam proses pengembangan sistem pembelajaran mesin, biasanya banyak tahapan yang perlu dilewati untuk memgembangkan sistem.
Dari proses pengumpulan data hingga proses \textit{delivery} dari sistem inferensi dilakukan berdasarkan kaidah-kaidah yang ada dalam MLOps.
Terdapat beberapa proses yang memakan waktu dalam proses pembuatannya, salah satunya adalah bagaimana sistem tersebut dibangun dari eksperimen-eksperimen yang dilakukan.

Salah satu yang menjadi perhatian penulis adalah proses konversi dari eksperimen menjadi sistem yang seutuhnya.
Dalam pembuatan sistem, banyak yang perlu diperhatikan dari sisi pengembangan perangkat lunak.
Namun, banyak hal di luar eksperimen model yang perlu diperhatikan dalam membuat layanan dengan teknologi pembelajaran mesin.

Dalam makalah ini, penulis akan membahas terkait hal-hal apa saja yang bisa diterapkan dalam proses pengembangan sistem pembelajaran mesin sehingga proses tersebut dapat dikerjakan dengan lebih efisien.
Khususnya, akan dibahas terkait teknologi yang bisa dikembangkan untuk melakukan konversi dari eksperimen-eksperimen yang dibuat menjadi satu sistem yang koheren dan \textit{production-ready}.