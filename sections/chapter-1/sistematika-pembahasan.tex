\section{Sistematika Pembahasan}

Dalam Bab I, akan dibahas terkait latar belakang dan tujuan dari makalah ini.
Selain itu, juga dibahas terkait masalah-masalah yang akan dibahas dan batasan dari masalah yang akan dibahas.
Bab I juga berisi bagaimana makalah ini akan disusun, serta metode penyelesaian dari masalah yang dibahas.

Bab II akan membahas terkait bahas teori yang dibutuhkan untuk menyelesaikan masalah serta mencari kemungkinan masalah yang ada.
Bahasan utama meliputi MLOps, DevOps, optomatisasi, serta metode umum dalam melakukan model serving dalam sebuah \textit{pipeline} MLOps.
Bab ini akan menekankan pada hal-hal apa saja yang menjadi dasar dari pengembangan kakas ini.

Bab III akan membahas proposal solusi untuk menyelesaikan masalah yang dibahas pada bagian sebelumnya.
Mulai dari analisis lanjut terkait permasalahan hingga bagaimana solusi dapat dirancang dan diimplementasikan.
Bab ini juga akan membahas mengenai hal-hal penting yang harus dicakup dari rancangan konfigurasi kakas yang perlu dibuat, serta bagaimana konfigurasi tersebut memetakan kode hasil buatan kakas.

Bab IV akan membahas mengenai teknis implementasi kakas.
Mulai dari bagaimana kakas ini diimplementasikan hingga studi kasus yang digunakan untuk menguji kakas ini.
Studi kasus yang dibahas merupakan studi kasus yang penulis anggap relevan dengan pembahasan dalam makalah ini.

Bab V akan menyimpulkan hal yang didapat berdasarkan hasil pengujian dan implementasi kakas.
Selain dari kesimpulan, bab ini akan membahas potensi pengembangan lanjut untuk kakas ini.