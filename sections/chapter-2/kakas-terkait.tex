\section{Kakas MLOps Terkait}

Dalam bagian ini, akan dibahas mengenai kakas atau projek yang dibuat untuk membantu proses MLOps.
Selain membahas kakas yang terkait dengan tugas akhir ini, akan dibahas sekilas mengenai kakas yang umum digunakan dalam proses pembangunan sistem pemelajaran mesin.
Hal ini dilakukan untuk lebih mengetahui kondisi kakas dan tren pembangunan sistem pemelajaran mesin saat ini.

\subsection{Kakas \textit{Versioning} Model Pemelajaran Mesin}

Di masa lampau, kebutuhan untuk sistem berbasis teknologi pemelajaran mesin tidak setinggi di masa kini. 
Perusahaan-perusahaan raksasa seperti Google membuat suatu \textit{framework} internal untuk kebutuhan mereka saat itu (\cite{mlops}).
Seiring meningkatnya keperluan untuk MLOps dalam dunia industri, kakas-kakas MLOps mulai bermunculan.

Kubeflow adalah salah satu kakas yang dikembangkan dengan tujuan untuk mempermudah proses \textit{deployment} pada sistem pemelajaran mesin dengan mudah (\cite{k8s}).
Seperti namanya, Kubeflow dibuat untuk melakukan \textit{deployment} di atas Kubernetes.
Kubeflow memiliki fitur yang lengkap, seperti adanya TensorFlow dan kakas-kakas pemelajaran mesin lainnya, kakas-kakas untuk melakukan eksperimen, menyimpan \textit{hyperparameter}, dan melakukan \textit{model versioning}, serta beberapa aplikasi yang digunakan untuk melakukan monitoring di Kubernetes juga, seperti Prometheus.
Dengan dibangunnya Kubeflow di atas Kubernetes, hal itu membuat melakukan \textit{deployment} pada \textit{platform-platform} berbasis \textit{cloud} menjadi mudah.
Dalam MLOps, Kubeflow ini membantu dalam tahapan \textit{model engineering} dan \textit{code engineering}.

Dalam Kubeflow, terdapat suatu sistem yang menggunakan \textit{pipeline} serupa dengan CI/CD, namun dengan konteks pengunaan yang berbeda.
Pipeline pada Kubeflow dapat membentuk sebuah graf yang masing-masing langkah atau \textit{node}-nya mendefinisikan parameter masukan dan keluaran yang terdapat pada tiap langkahnya.
Tiap langkah merepresentasikan suatu proses terhadap data yang dilakukan lewat Docker yang membungkus suatu kode Python.
Pendekatan \textit{pipeline} ini secara umum fleksibel dan dapat menjadi potensi pengembangan untuk MLOps.

Selain dari kemampuannya untuk melakukan \textit{model versioning}, Kubeflow juga memiliki kemampuan untuk melakukan \textit{deployment} terhadap model.
Model hasil eksperimen yang dilakukan akan di-\textit{deploy}.
Fitur yang ditawarkan kubeflow membantu proses pembangunan sistem, akan tetapi sistem yang dihasilkan tidak fleksibel untuk diubah untuk keperluan yang berbeda.
Selain itu, biasanya kakas ini digunakan untuk lingkungan \textit{production} berskala besar.

\subsection{Kakas \textit{Serving} Model Pemelajaran Mesin}

Dalam proses melakukan \textit{deployment} terhadap model yang sudah menjadi sistem secara utuh, seringkali dibutuhkan suatu kakas untuk membantu proses tersebut.
Salah satu kakas terkait yang dapat membantu proses \textit{deployment} adalah Tensorflow Serving (\href{https://github.com/tensorflow/serving}{https://github.com/tensorflow/serving}).

Kakas ini adalah kakas yang digunakan untuk membantu proses \textit{deployment} pada model pemelajaran mesin yang dibuat dengan Tensorflow.
Secara sederhana, kakas ini menyediakan kemampuan untuk membungkus model Tensorflow yang ada lewat metode komunikasi yang umum seperti gRPC dan REST.\@
Cara menjalankannya juga cukup mudah, yaitu hanya dengan menggunakan Docker untuk melakukan \textit{mounting} terhadap model.

Kakas ini cukup membantu dalam proses \textit{deployment} yang cepat untuk mendapatkan prototipe dari layanan yang seutuhnya.
Dalam MLOps, Tensorflow Serving ini membantu dalam tahapan \textit{code engineering}.
Kakas ini hanya berfokus pada cara model dapat diakses oleh metode komunikasi umum.
Terdapat beberapa hal yang tidak diatasi oleh kakas ini dari sisi pengembangan sistem, seperti pembuatan dokumentasi API dapat dibentuk berdasarkan masukan yang ada dan ketiadaan fitur untuk menggunakan model sebagai \textit{dependency} dari sistem.

