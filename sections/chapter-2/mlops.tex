\section{MLOps}

\subsection{Definisi MLOps}

MLOps adalah bagian yang lebih spesifik dari DevOps. 
Seperti yang telah dibahas pada bagian sebelumnya, DevOps adalah suatu metodologi dan prinsip yang diterapkan untuk membuat proses pengembangan sistem menjadi lebih cepat dan efisien. 
MLOps adalah metodologi yang menerapkan DevOps untuk proses-proses yang ada pada pengembangan model machine learning.

Dalam MLOps, terdapat proses-proses tertentu yang spesifik untuk domain pengembangan \textit{pipeline} ML.
Seperti dilansir dari artikel oleh~\cite{mlops}, MLOps ini merupakan bidang yang masih baru karena baru-baru ini teknologi kecerdasan buatan mulai diterapkan dalam perusahaan.
Tujuan dari MLOps adalah untuk melakukan manajemen dari eksperimen dalam proses pengembangan sistem intelijen.
Saat ini, metodologi MLOps juga sudah diterapkan pada berbagai perusahaan besar di dunia, seperti Facebook, Google, AirBnB, dan Uber.

\subsection{Contoh Kakas MLOps}

Di masa lampau, kebutuhan untuk sistem intelijen tidak setinggi di masa kini. 
Oleh karena itu, seperti yang ditulis oleh~\cite{mlops}, perusahaan-perusahaan raksasa seperti Google membuat suatu \textit{framework} internal untuk kebutuhan mereka saat itu.
Seiring meningkatnya keperluan untuk MLOps dalam dunia industri, kakas-kakas MLOps mulai bermunculan.

Kubeflow, seperti tertulis dalam dokumentasi resminya adalah salah satu kakas yang dikembangkan dengan tujuan untuk mempermudah proses \textit{deployment} pada sistem ML dengan mudah.
Seperti namanya, Kubeflow dibuat untuk melakukan \textit{deployment} di atas Kubernetes.
Kubeflow memiliki fitur yang lengkap, seperti adanya TensorFlow dan kakas-kakas ML lainnya, kakas-kakas untuk melakukan eksperimen, menyimpan \textit{hyperparameter}, dan melakukan \textit{model versioning}, serta beberapa aplikasi yang digunakan untuk melakukan monitoring di Kubernetes juga, seperti Prometheus.
Dengan dibangunnya Kubeflow di atas Kubernetes, hal itu membuat melakukan  \textit{deployment} pada \textit{platform-platform} berbasis \textit{cloud} menjadi mudah.

