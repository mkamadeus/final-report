\section{DevOps}

\subsection{Gambaran Umum DevOps}
\begin{figure}[h]
  \centering
  \includegraphics[width=0.8\textwidth]{devops.png}
  \caption{Mekanisme DevOps (Sumber:~\cite{devops})}
\end{figure}
DevOps adalah serangkaian filosofi, \textit{practices}, dan kakas yang bisa meningkatkan produktivitas perusahaan, baik secara internal maupun untuk mempercepat proses \textit{delivery} layanan untuk konsumen (\cite{devops}).
Dengan \textit{delivery} yang cepat, tentunya akan bisa berkompetisi dengan lebih efektif di pasar.
Terdapat lima bagian utama dari filosofi DevOps, yaitu build, test, release, plan, dan monitor.
Hal yang menjadi tujuan utama untuk DevOps adalah untuk mempercepat dalam proses-proses yang memakan waktu dengan menggunakan otomatisasi yang bisa dilakukan terhadap sistem.

\subsection{\textit{Practices} dalam DevOps}
Terdapat beberapa hal yang menjadi \textit{best practices} dalam DevOps. 
Dalam bagian ini, penulis akan membahas sedikit terkait masing-masing practices yang ada di dalam DevOps yang relevan dengan poin-poin yang ada pada makalah ini.

\subsubsection{Continuous Integration}
Continuous integration adalah filosofi yang diterapkan dengan tujuan memastikan integritas sistem.
Biasanya, continuous integration dicapai lewat menerapkan \textit{build} dan \textit{test} yang otomatis.
Continuous integration akan sangat menunjang TDD (\textit{Test Driven Development}) karena selalu dilakukan secara otomatis setiap kali ada perubahan yang dilakukan kepada sistem.
Dengan begitu, kualitas sistem, pencarian bug, dan penyelesaiannya bisa diselesaikan dengan lebih cepat.


\subsubsection{Continuous Delivery}
Continuous delivery adalah suatu filosofi dengan tujuan untuk memastikan \textit{delivery} dari sistem.
Dalam konteks ini, \textit{delivery} yang dimaksud adalah terkait melakukan \textit{build} dan \textit{deployment} ke \textit{production} secara otomatis.
Continuous delivery juga sangat erat hubungannya dengan continuous integration, yang terkadang bisa juga dilakukan dalam \textit{pipeline} yang sama.
Pengaturan \textit{deployment} juga fleksibel, tidak harus melulu untuk \textit{production} saja, walaupun tujuan awalnya untuk melakukan \textit{delivery} dengan cepat.

\subsubsection{Microservices}
Microservices adalah sebuah aristektur untuk sistem, di mana sebuah sistem besar dibuat atas dasar sistem-sistem kecil yang membentuk satu kesatuan besar.
Secara umum, microservices dapat memberi dampak positif bagi tim yang besar, karena pengembangan bisa dilakukan secara independen antar sistem.
Walau begitu, dari sudut pandang deployment tentunya tidak trivial.
Meskipun demikian, microservices akan mendukung \textit{scalability} dari sistem ke depannya, dan dengan filosofi DevOps lainnya aristektur ini sangat baik untuk diterapkan dalam lingkungan pengembangan dengan jumlah orang yang banyak. 

\subsubsection{Infrastructure as Code}
Infrastructure as Code adalah suatu filosofi di mana infrastruktur yang dibuat harus dapat disimpan sebagai program, dengan tujuan memudahkan replikasi dan manajemen infrastruktur sistem.
Seiring meningkatnya penggunaan sistem berbasis \textit{cloud}, penggunaan Infrastructure as Code juga meningkat karena alasan yang disebut sebelumnya.
Hal-hal yang bisa dimanajemen lewat infrastructure as code meliputi konfigurasi infrastruktur dan policy atau aturan terhadap infrastruktur.
Dengan kakas-kakas yang ada sekarang, konfigurasi yang ada bisa direalisasikan menjadi infrastruktur yang sesungguhnya.

\subsubsection{Monitoring dan Logging}
Dalam DevOps, Monitoring dan Logging terhadap sistem menjadi suatu hal yang patut diperhatikan.
Dengan sistem yang terus berkembang, sistem perlu diawasi untuk memastikan kelancaran dari sistem.
 
