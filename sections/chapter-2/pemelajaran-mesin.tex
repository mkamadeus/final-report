\section{Pemelajaran Mesin}

Pemelajaran mesin atau yang lebih dikenal dengan istilah \textit{machine learning} adalah ilmu yang merupakan bagian dari algoritma pemrograman yang membolehkan suatu mesin atau komputer berkembang seiring dengan meningkatnya pengalaman (\cite{mitchell_1997}).
Pengalaman seringkali merujuk kepada data yang diberikan kepada program.
Pada dasarnya, pemelajaran mesin adalah suatu teknik yang memanfaatkan dan algoritma optimasi untuk menghasilkan suatu bentuk prediksi atau keluaran tertentu.

Terdapat beberapa teknik dan algoritma pemelajaran mesin untuk menyelesaikan suatu masalah dalam beberapa domain permasalahan.
Saat ini, terdapat beberapa teknik yang umum digunakan, yaitu: regresi, klasifikasi, pengelompokan/\textit{clustering}, dan lain-lain.
Teknik-teknik pemelajaran mesin baru terus menerus dikembangkan secara umum atau untuk domain permasalahan khusus.
Bagian ini hanya akan membahas secara umum saja mengenai definisi \textit{machine learning} agar lebih dipahami konteks dari permasalahan yang diangkat pada tugas akhir ini.

Saat ini, penggunaan teknologi pemelajaran mesin meluas seiring dengan meningkatnya kemampuan komputasi dan jumlah data yang terdapat dan dikumpulkan.
Oleh karena itu, sistem dengan teknologi pemelajaran mesin semakin berkembang dan digunakan diberbagai konteks permasalahan maupun di proses bisnis.
Rincian lanjut mengenai bagaimana sistem dengan teknologi pemelajaran mesin dikembangkan akan dibahas pada bagian-bagian berikutnya.
