\section{Model Serving}\label{chap:model-serving}

Dalam makalah ini akan dibahas secara lebih mendalam terkait Model Serving.
Hal-hal yang akan dibahas meliputi bagaimana metode-metode yang ada dalam melakukan serving dan teknologi apa yang digunakan untuk melakukan \textit{deployment} model tersebut.

Terdapat lima jenis \textit{serving pattern} yang umum digunakan untuk membuat layanan dari model (\cite{mlopsorg}):
\begin{enumerate}
  \item Model-as-Service
  \item Model-as-Dependency
  \item Precompute Serving
  \item Model-on-Demand
  \item Hybrid Serving
\end{enumerate}

\subsection{Model-as-Service}

\begin{figure}
  \centering
  \includegraphics[width=0.7\textwidth]{02-model-as-service.jpg}
  \caption{Ilustrasi Model-as-Service (Sumber:~\cite{book-handsonml})}\label{fig:model-as-service}
\end{figure}

\textit{Serving pattern} ini adalah metode yang paling sederhana.
Pada dasarnya, model yang sudah ada akan dibungkus dengan sebuah interface agar dapat dilakukan RPC terhadap model tersebut (lihat Gambar~\ref{fig:model-as-service}).
Teknologi yang umum digunakan biasanya seperti REST API dan gRPC, namun metode-metode RPC lainnya bisa juga digunakan.

\subsection{Model-as-Dependency}

\begin{figure}
  \centering
  \includegraphics[width=0.7\textwidth]{02-model-as-dependency.jpg}
  \caption{Ilustrasi Model-as-Dependency (Sumber:~\cite{book-handsonml})}\label{fig:model-as-dependency}
\end{figure}

\textit{Serving pattern} serupa dengan Model-as-Service, tetapi perbedaan utamanya adalah di mana model tersebut menjadi dependency dari layanan yang dibuat secara internal, bukan melakukan \textit{deployment} secara terpisah untuk model tersebut (lihat Gambar~\ref{fig:model-as-dependency}).
\textit{Pattern} ini akan digunakan apabila model akan digunakan sebagai satu bagian dari layanan yang lebih besar.

\subsection{Precompute Serving}

\begin{figure}
  \centering
  \includegraphics[width=0.7\textwidth]{02-precompute-serving-pattern.jpg}
  \caption{Ilustrasi Precompute Serving Pattern (Sumber:~\cite{book-handsonml})}\label{fig:precompute-serving}
\end{figure}

Sesuai namanya, model akan digunakan untuk melakukan prekomputasi tanpa langsung menggunakan modelnya dalam sistem (lihat Gambar~\ref{fig:precompute-serving}).
Hasil dari prekomputasi yang dilakukan biasanya akan disimpan pada suatu basis data, yang nantinya akan digunakan oleh layanan tertentu.
Sehingga, model tidak perlu dijalankan terus menerus; hanya perlu dijalankan bila diperlukan atau secara berkala saja.

\subsection{Model-on-Demand}

\begin{figure}
  \centering
  \includegraphics[width=0.7\textwidth]{02-model-on-demand.jpg}
  \caption{Ilustrasi Model-on-Demand (Sumber:~\cite{book-handsonml})}\label{fig:model-on-demand}
\end{figure}

Pada dasarnya, \textit{pattern} ini memiliki karakteristik yang serupa dengan Model-as-Service dengan perbedaan dalam infrastruktur pembentuk sistem.
Perbedaan utamanya adalah dalam Model-as-Service, request dijalankan secara sinkron.
Dalam \textit{pattern} Model-on-Demand, terdapat penggunaan \textit{broker}  sehingga request terhadap model bisa dibuat asinkron (lihat Gambar~\ref{fig:model-on-demand}).
Waktu prediksi juga bisa diatur lewat broker atau model yang digunakan.

\subsection{Hybrid Serving}

\begin{figure}
  \centering
  \includegraphics[width=0.7\textwidth]{02-federated-learning.jpg}
  \caption{Ilustrasi Federated Learning (Sumber:~\cite{book-handsonml})}\label{fig:federated-learning}
\end{figure}

Hybrid Serving, atau biasa dikenal dengan istilah Federated Learning adalah teknik yang memanfaatkan beberapa perangkat untuk proses \textit{training}-nya.
Tidak hanya melakukan \textit{training} secara terpusat, model yang sudah jadi akan didistribusikan pada perangkat-perangkat berbeda, dan dari sana data baru akan dikumpulkan (lihat Gambar~\ref{fig:federated-learning}).
Data yang dikumpulkan akan dikirim ke sebuah central store untuk membuat model yang lebih mutakhir.
