\section{Otomatisasi}\label{automation}

Dalam bagian ini, akan dibahas terkait prinsip dan konsep yang secara umum diterapkan dalam melakukan otomatisasi sistem. Selain itu, akan dibahas terkait aspek apa saja dari sebuah sistem yang diotomatisasi. Hal yang akan dibahas pada bagian ini adalah hal-hal yang sudah terbukti efektif untuk diotomatisasi berdasarkan\cite{beyer2016site}.

\subsection{Alasan Melakukan Otomatisasi}
Otomatisasi di masa ini menjadi hal yang diusahakan oleh tim pengembang sistem dan perangkat lunak, terutama pada tim yang besar. Walaupun begitu, terkadang otomatisasi menjadi hal yang dikesampingkan dalam proses pengembangan, yang seharusnya dapat mengingkatkan produktivitas pengembangan.

Pada sub bagian ini, penulis akan membahas terkait beberapa alasan untuk melakukan otomatisasi. Alasan-alasan berikut dikutip dari~\cite{beyer2016site}, yaitu sebagai berikut:

\begin{itemize}
  \item Menetapkan konsistensi dalam melakukan pekerjaan yang repetitif
  \item Memusatkan titik kesalahan dalam pengembangan sistem
  \item Mempercepat proses perbaikan sistem
  \item Memperkecil waktu aksi untuk melakukan suatu kegiatan terkait infrastruktur
  \item Menghemat waktu secara umum
\end{itemize}

Dalam otomatisasi, tentunya konsistensi bisa ditetapkan secara internal dengan asumsi faktor eksternal di luar dari kode sistem tidak ada masalah. Otomatisasi yang seharusnya dilakukan hanya melakukan satu hal saja secara konsisten. Bila terjadi masalah di luar fungsionalitas internal sistem, kemungkinan besar masalah tersebut berasal dari faktor eksternal.

Dengan melakukan otomatisasi juga, secara umum tim-tim menggunakan satu platform. Tentu tujuannya untuk mempermudah tim dalam melakukan otomatisasi, sehingga semuanya serba terpusat pada platform tersebut. Hal ini membawa dampak positif juga, karena semua masalah bisa ditelusuri dari platform tersebut.

Perbaikan sistem menjadi hal umum yang dilakukan di setiap proses pengembangan. Oleh karena itu, otomatisasi tentunya dapat membantu proses perbaikan agar lebih efisien, sehingga waktu ynag tersisa bisa digunakan untuk tugas atau kegiatan lain yang lebih produktif untuk pengembangan sistem.

Seperti dituliskan sebelumnya, mengurangi kegiatan yang repetitif dengan otomatisasi bisa menambah waktu untuk pengembangan sistem. Dalam kata lain, aksi-aksi tertentu juga dapat dilakukan lebih cepat lewat proses otomatisasi ini. Tentu saja, salah satu tujuan utamanya untuk menghemat waktu dari proses pengembangan ini. Walaupun mungkin tidak begitu terasa di awal dan mungkin malah menambah \textit{overhead} dalam proses awal pengembangan, semakin lama nilai yang didapatkan dari otomatisasi semakin besar.

\subsection{Aspek Otomatisasi}
Dalam bagian ini tidak akan dibahas secara menyeluruh terkait aspek apa saja yang dapat diotomatisasi, karena pada dasarnya apapun bisa diotomatisasi dalam sebuah sistem, namun dampak yang diberikan bisa jadi belum tentu positif. Oleh karena itu, pada bagian ini aspek-aspek yang sudah terbukti berdasarkan~\cite{beyer2016site} dan beberapa pengalaman pribadi penulis.

Mengutip dari~\cite{beyer2016site}, beberapa hal yang bisa dilakukan seputar otomatisasi sistem adalah sebagai berikut:
\begin{itemize}
  \item Pembuatan user account baru
  \item Melakukan turnup dan turndown pada sebuah layanan di cluster
  \item Melakukan rilis versi baru
  \item Menginstall software dengan versi yang lebih baru
\end{itemize}

Tentunya, masih banyak hal yang bisa diotomatisasi. Daftar tersebut hanya daftar singkat dari apa yang bisa diotomatisasi. Namun, perlu kita ingat bahwa belum tentu otomatisasi yang dilakukan efektif, sehingga kita perlu mengujinya secara langsung untuk melihat apakah \textit{feasible} untuk melakukan hal tersebut.

Salah satu hal yang perlu dipertimbangkan dalam membuat sebuah \textit{pipeline} dalam mengotomatisasi sistem adalah masalah-masalah umum yang sudah dibahas pada bagian~\ref{problems}. Masalah-masalah tersebut bisa menjadi pedoman untuk membuat otomatisasi pada sistem lewat teknologi-teknologi yang ada.

