\section{Otomatisasi}\label{automation}

Dalam bagian ini, akan dibahas prinsip dan konsep yang secara umum diterapkan dalam melakukan otomatisasi sistem.
Selain itu, akan dibahas terkait aspek apa saja dari sebuah sistem yang diotomatisasi.
Hal yang akan dibahas pada bagian ini adalah hal-hal yang sudah terbukti efektif untuk diotomatisasi.

\subsection{Alasan Melakukan Otomatisasi}
Otomatisasi di masa ini menjadi hal yang diusahakan oleh tim pengembang sistem dan perangkat lunak, terutama pada tim yang besar.
Walaupun begitu, kadang otomatisasi menjadi hal yang dikesampingkan dalam proses pengembangan, padahal ada kemungkinan otomatisasi dapat mengingkatkan produktivitas pengembangan.

Pada sub bagian ini, penulis akan membahas terkait beberapa alasan untuk melakukan otomatisasi (\cite{beyer2016site}), yaitu sebagai berikut:

\begin{itemize}
  \item Menetapkan konsistensi dalam melakukan pekerjaan yang repetitif
  \item Memusatkan titik kesalahan dalam pengembangan sistem
  \item Mempercepat proses perbaikan sistem
  \item Memperkecil waktu aksi untuk melakukan suatu kegiatan terkait infrastruktur
  \item Menghemat waktu secara umum
\end{itemize}

Dalam otomatisasi, tentunya konsistensi dapat ditetapkan secara internal dengan asumsi faktor eksternal di luar dari kode sistem tidak ada masalah.
Otomatisasi yang seharusnya dilakukan hanya melakukan satu hal saja secara konsisten.
Bila terjadi masalah di luar fungsionalitas internal sistem, kemungkinan besar masalah tersebut berasal dari faktor eksternal.

Dengan melakukan otomatisasi juga, secara umum tim-tim menggunakan satu platform.
Tentu tujuannya untuk mempermudah tim dalam melakukan otomatisasi, sehingga semuanya serba terpusat pada platform tersebut.
Hal ini membawa dampak positif juga, karena semua masalah dapat ditelusuri dari platform tersebut.

Perbaikan sistem menjadi hal umum yang dilakukan di setiap proses pengembangan.
Oleh karena itu, otomatisasi tentunya dapat membantu proses perbaikan agar lebih efisien, sehingga waktu yang tersisa dapat digunakan untuk tugas atau kegiatan lain yang lebih produktif untuk pengembangan sistem.

Seperti dituliskan sebelumnya, mengurangi kegiatan yang repetitif dengan otomatisasi dapat menambah waktu untuk pengembangan sistem.
Dalam kata lain, aksi-aksi tertentu juga dapat dilakukan lebih cepat lewat proses otomatisasi ini.
Tentu saja, salah satu tujuan utamanya untuk menghemat waktu dari proses pengembangan ini.
Walaupun mungkin tidak begitu terasa di awal dan mungkin malah menambah \textit{overhead} dalam proses awal pengembangan, semakin lama nilai yang didapatkan dari otomatisasi semakin besar.

\subsection{Aspek Otomatisasi}
Dalam bagian ini tidak akan dibahas secara menyeluruh terkait aspek apa saja yang dapat diotomatisasi, karena pada dasarnya apapun dapat diotomatisasi dalam sebuah sistem, namun belum tentu berdampak secara signifikan terhadap sistem.
Oleh karena itu, bagian ini akan membahas aspek-aspek yang sudah terbukti dapat diotomatisasi dan memberi dampak terhadap sistem.

Beberapa hal yang dapat dilakukan seputar otomatisasi sistem adalah sebagai berikut (\cite{beyer2016site}):
\begin{itemize}
  \item Pembuatan \textit{user account} baru.
  \item Melakukan \textit{turnup} dan \textit{turndown} pada sebuah layanan di kluster.
  \item Melakukan rilis versi baru.
  \item Melakukan instalasi perangkat lunak dengan versi yang lebih baru
\end{itemize}

Tentunya, masih banyak hal yang dapat diotomatisasi di luar hal yang disebutkan di atas.
Daftar tersebut hanya daftar singkat dari apa yang dapat diotomatisasi dari sistem secara umum. 
Perlu kita ingat bahwa otomatisasi yang dilakukan belum tentu efektif, sehingga kita perlu mengkaji lebih lanjut untuk mengetahui seberapa\textit{feasible} dan efektif untuk mengotomatisasikan suatu hal.
Masalah-masalah yang ada dalam sebuah sistem dapat menjadi pedoman untuk membuat otomatisasi pada sistem lewat teknologi-teknologi yang ada.

\section{Potensi Otomatisasi dalam MLOps}

Seperti yang telah dibahas pada bagian sebelumnya, MLOps terdiri dari tiga tahapan utama, yaitu Data Engineering, Model Engineering, dan Code Engineering.
Dalam praktiknya, bagian yang merupakan \textit{toil} dalam ketiga proses ini berada paling banyak dalam tahap Data Engineering.
\textit{Preprocessing} dan eksplorasi data umumnya perlu butuh waktu yang lama karena melibatkan beberapa domain dan beberapa faktor lainnya.
Perlu metodologi tertentu untuk membantu dalam proses preprocessing data baik secara infrastruktur dan metode pemrosesannya.
Contohnya, dalam Kubeflow terdapat fitur melakukan pipelining untuk melakukan proses-proses tertentu dalam langkah yang berbeda-beda.

Selain tahap tersebut, dalam tahap lain juga terdapat potensi untuk melakukan otomatisasi, seperti dalam proses pembuatan sistem inferensinya.
Seperti yang disinggung sebelumnya, sudah tersedia kakas bantuan untuk melakukan \textit{serving} terhadap model.
Masih terdapat ruang untuk melakukan pengembangan kakas yang dapat meningkatkan produktivitas pembuatan sistem.