\section{Kesimpulan}

Berikut adalah beberapa poin penting yang ditarik dari percobaan kakas ini.

\begin{enumerate}
    \item Belum ada \textit{best practice} untuk alur konversi eksperimen menjadi sistem.
    \item Kakas \monospace{myx} berhasil diimplementasikan dan dapat diterapkan dalam sebuah \textit{workflow} pembangunan sistem pemelajaran mesin dengan tujuan utamanya adalah untuk membantu dan mempercepat proses konversi eksperimen menjadi sistem melalui teknik pembangkitan kode/\textit{code-generation}.
    \item Dalam kakas yang dibangun, diimplementasikan beberapa metode untuk membantu konversi sistem, yaitu konfigurasi
    \item Terdapat 4 kasus yang berhasil diuji untuk kakas ini yang merupakan campuran dari beberapa kasus, yaitu: data tabular dengan \textit{shallow learning}, data tabular dengan \textit{deep learning}, data citra dengan \textit{deep learning}, dan data citra dengan \textit{deep learning} lewat metode \textit{transfer learning}.
    \item Kakas ini dapat diadaptasikan untuk kasus lainnya seperti format masukan yang berbeda untuk permasalahan di luar data tabular dan citra.
\end{enumerate}
