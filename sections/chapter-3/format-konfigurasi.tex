\section{Format Konfigurasi}

Seperti yang disinggung pada bagian sebelumnya, kakas ini akan menerima masukan berupa \textit{markup file} yang menjadi konfigurasi dari sistem.
\textit{File} ini tentunya mengandung pengaturan yang diperlukan dalam membangun suatu kode sistem.
Beberapa hal yang terdapat dari konfigurasi berdasarkan pertimbangan penulis untuk menyelesaikan masalah adalah sebagai berikut:

\begin{enumerate}
	\item Format masukan
	\item Format keluaran
	\item Format model
	\item Alur pemrosesan
	\item \textit{Interface} yang disediakan
\end{enumerate}

Sebagai catatan, format masukan dan keluaran yang dimaksudkan adalah untuk pengunaan sistem secara \textit{interface} dan bukan untuk model yang menjadi bagian dari sistem.
Dalam bagian ini, akan dibahas mengenai komponen untuk 

\subsection{Konfigurasi Format Masukan}

Sistem pemelajaran mesin saat ini memiliki banyak bentuk masukan.
Salah satu bentuk masukan yang umum digunakan adalah bentuk yang berupa tabel atau tabular.
Format masukan perlu didefinisikan untuk membantu dalam proses pembuatan kode khususnya terkait dengan \textit{interface}.
Berikut adalah contoh konfigurasi untuk bagian masukan sistem yang berupa data tabular:

\yamlcode{resources/files/configurations/input_basic.yml}

Mengingat jenis masukan bisa beragam, tidak menutup kemungkinan masukan dengan bentuk lain bisa digunakan dalam model, contohnya seperti masukan yang berupa gambar.
Masukan gambar memiliki aturan-aturan seperti layaknya data tabular, seperti panjang dan lebar dari sebuah gambar.
Berikut adalah contoh konfigurasi untuk masukan yang berupa data gambar:

\yamlcode{resources/files/configurations/input_image.yml}

Selain dari format masukan yang beragam, terkadang masukan kepada sistem bisa berbeda dengan masukan pada model.
Alasan dari perbedaan tersebut adalah terdapat pemrosesan data yang dilakukan oleh sistem untuk membuat masukan pada sistem lebih intuitif.
Oleh karena itu, konfigurasi dari format masukan juga perlu menangani hal tersebut.
Berikut adalah contoh konfigurasi untuk masukan yang perlu dilakukan pemrosesan data:

\yamlcode{resources/files/configurations/input_preprocessed.yml}

\subsection{Konfigurasi Format Keluaran}
Serupa dengan bagaimana format masukan didefinisikan, format keluaran perlu didefinisikan juga.
Format keluaran tidak memiliki variasi lebih banyak dari format masukan.
Format keluaran bergantung dari domain permasalahan serta teknik-teknik pembelajaran mesin yang digunakan.
Misalnya, dalam permasalahan klasifikasi format keluaran akan berupa sekumpulan prediksi dari model yang mungkin disertai dengan derajat kepercayaan dari prediksi tersebut.
Permasalahan regresi juga akan mengembalikan suatu nilai numerik.
Tidak sebatas dalam teknik-teknik \textit{supervised} saja, teknik \textit{unsupervised} seperti \textit{clustering} yang memiliki keluaran yang serupa dengan klasifikasi.
Berikut adalah contoh konfigurasi untuk keluaran sederhana:

\yamlcode{resources/files/configurations/output_basic.yml}

\subsection{Konfigurasi Format Model}
Eksperimen untuk membuat model pembelajaran mesin menerapkan penggunaan banyak kakas-kakas bantuan yang umum digunakan, seperti Tensorflow, SKLearn, Catboost, dan banyak kakas-kakas lainnya.
Akibat kakas-kakas yang beragam, secara umum model yang telah dilatih akan disimpan dalam sebuah \textit{file} dengan format ekstensi tertentu.
Format yang penulis pilih adalah format ONNX (Open Neural Network Exchange) karena memiliki banyak integrasi dengan kakas-kakas yang bermacam-macam jenisnya.
ONNX juga memiliki kemampuan untuk melakukan inferensi dengan cara yang seragam untuk kakas-kakas yang berbeda jenis.

Walaupun begitu, tidak menutup kemungkinan untuk konfigurasi tidak menggunakan format lain.
Misalnya, pada kakas Tensorflow disediakan sebuah cara untuk menyimpan model dalam sebuah file lewat \mintinline{python}{save_model()}.
Oleh karena itu, konfigurasi format model harus bisa menerima jenis model yang berbeda-beda.
Berikut ini adalah contoh konfigurasi sederhana untuk format model

\subsection{Konfigurasi Alur Pemrosesan}
Secara umum, terkadang perlu tahapan-tahapan tertentu dari masukan yang diberikan terhadap sistem.
Misalnya, dalam data tabular kadang dilakukan proses \textit{scaling} terhadap data atau melakukan reduksi dimensi dengan menggunakan teknik seperti PCA dan LDA.
Masukan berupa gambar kadang diperlukan untuk melakukan tahap \textit{image processing} untuk memastikan gambar yang dimasukkan cocok untuk sistem.
Kakas ini akan menerima konfigurasi untuk melakukan hal-hal tersebut lewat modul-modul yang didefinisikan dalam konfigurasi.

\subsection{Konfigurasi \textit{Interface}}
Sistem secara umum akan menggunakan salah satu atau kedua dari \textit{interface}, yaitu REST atau gRPC.
Dengan mempertimbangkan kedua hal tersebut, kakas ini akan berfokus pada kedua \textit{interface} tersebut.
Kode dari sistem yang dibangkitkan tentunya dapat dimodifikasi sesuai kebutuhan dari sistem bila diperlukan perubahan terhadap \textit{interface}.
