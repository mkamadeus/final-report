\section{Format Konfigurasi}
% chktex-file 36

Seperti yang disinggung pada bagian sebelumnya, kakas ini akan menerima masukan berupa \textit{markup file} yang menjadi konfigurasi dari sistem.
\textit{File} ini mengandung pengaturan yang diperlukan dalam membangun suatu kode sistem.
Beberapa hal yang terdapat dari konfigurasi berdasarkan pertimbangan penulis untuk menyelesaikan masalah adalah sebagai berikut:

\begin{enumerate}
	\item Format masukan
	\item Format keluaran
	\item Format model
	\item Alur pemrosesan
	\item \textit{Interface} yang disediakan
\end{enumerate}

Sebagai catatan, format masukan dan keluaran yang dimaksudkan adalah untuk pengunaan sistem secara \textit{interface} dan bukan untuk model yang menjadi bagian dari sistem.
Dalam bagian ini, akan dibahas mengenai masing-masing komponen yang terdapat dalam konfigurasi secara singkat.
Rancangan konfigurasi ini adalah bayangan penulis dalam membuat konfigurasi yang intuitif serta dapat dikembangkan lebih lanjut.
Bagaimana kakas akan menggunakan konfigurasi ini akan dibahas pada bagian berikutnya.

\subsection{Konfigurasi Format Masukan}\label{section:03-input-format}

Sistem pemelajaran mesin saat ini memiliki banyak bentuk masukan.
Salah satu bentuk masukan yang umum digunakan adalah bentuk yang berupa tabel atau tabular.
Format masukan perlu didefinisikan untuk membantu dalam proses pembuatan kode khususnya terkait dengan \textit{interface}.
Kode~\ref{listing:1} merupakan contoh konfigurasi untuk bagian masukan sistem yang berupa data tabular.

\begin{code}
	\yamlcode{resources/files/configurations/input_basic.yml}
	\caption{Contoh spesifikasi masukan tabular untuk kakas}\label{listing:1}
\end{code}

Untuk data tabular, terdapat beberapa parameter yang bisa dikonfigurasi.
Parameter ``name'' pada konfigurasi menjadi nama masukan pada file JSON yang menjadi masukan sistem.
Parameter ``type'' mendefinisikan tipe masukan data kepada sistem.
Parameter ``target'' mendefinisikan ke kolom mana masukan ini akan digunakan dalam masukan model.

Mengingat jenis masukan dapat beragam, tidak menutup kemungkinan masukan dengan bentuk lain dapat digunakan dalam model, contohnya seperti masukan yang berupa citra.
Masukan citra memiliki aturan-aturan seperti layaknya data tabular, seperti panjang dan lebar dari sebuah citra.
Kode~\ref{listing:2} merupakan contoh konfigurasi untuk masukan yang berupa data citra.

\begin{code}
	\yamlcode{resources/files/configurations/input_image.yml}
	\caption{Contoh spesifikasi masukan citra untuk kakas}\label{listing:2}
\end{code}

Serupa pada format data tabular, data citra juga memiliki beberapa parameter yang bisa diatur.
Misalnya, parameter ``channel'' yang mendefinisikan berapa \textit{channel} yang terdapat pada citra.
Terdapat juga parameter yang mendefinisikan ukuran citra yang dimasukkan ke dalam model.

Selain dari format masukan yang beragam, terkadang masukan kepada sistem dapat berbeda dengan masukan pada model seperti halnya dalam beberapa kasus di permasalahan data tabular.
Alasan dari perbedaan tersebut adalah terdapat pemrosesan data yang dilakukan oleh sistem untuk membuat masukan pada sistem lebih intuitif.
Tidak semua masukan dapat diketahui nilai yang perlu dimasukan ke dalam model tanpa melalui pemrosean yang sama seperti dalam eksperimen.

Oleh karena itu, konfigurasi dari format masukan juga perlu menangani hal tersebut.
Kolom yang perlu dilakukan pemrosesan tambahan ditandai untuk dilihat lebih lanjut dalam bagian pemrosesan data.
Kode~\ref{listing:3} merupakan contoh konfigurasi untuk masukan yang ditandai untuk dilakukan pemrosesan data.

\begin{code}
	\yamlcode{resources/files/configurations/input_preprocessed.yml}
	\caption{Contoh spesifikasi masukan yang perlu pemrosesan lanjut}\label{listing:3}
\end{code}

\subsection{Konfigurasi Format Keluaran}\label{section:03-output-format}

Serupa dengan bagaimana format masukan didefinisikan, format keluaran perlu didefinisikan juga.
Format keluaran tidak memiliki variasi lebih banyak dari format masukan.
Format keluaran bergantung dari domain permasalahan serta teknik-teknik pembelajaran mesin yang digunakan.
Misalnya, dalam permasalahan klasifikasi format keluaran akan berupa sekumpulan prediksi dari model yang mungkin disertai dengan derajat kepercayaan dari prediksi tersebut.
Permasalahan regresi juga akan mengembalikan suatu nilai numerik.
Tidak sebatas dalam teknik-teknik \textit{supervised} saja, teknik \textit{unsupervised} seperti \textit{clustering} yang memiliki keluaran yang serupa dengan klasifikasi.
Kode~\ref{listing:4} merupakan contoh konfigurasi untuk keluaran sederhana.

\begin{code}
	\yamlcode{resources/files/configurations/output_basic.yml}
	\caption{Contoh spesifikasi keluaran untuk sistem}\label{listing:4}
\end{code}

\subsection{Konfigurasi Format Model}\label{section:03-model-format}

Eksperimen untuk membuat model pembelajaran mesin menerapkan penggunaan banyak kakas-kakas bantuan yang umum digunakan, seperti Tensorflow, SKLearn, Catboost, dan banyak kakas-kakas lainnya.
Akibat kakas-kakas yang beragam, secara umum model yang telah dilatih akan disimpan dalam sebuah \textit{file} dengan format ekstensi tertentu.
Format yang penulis pilih adalah format ONNX (Open Neural Network Exchange) karena memiliki banyak integrasi dengan kakas-kakas yang bermacam-macam jenisnya.
ONNX juga memiliki kemampuan untuk melakukan inferensi dengan cara yang seragam untuk kakas-kakas yang berbeda jenis.

Walaupun begitu, tidak menutup kemungkinan untuk konfigurasi tidak menggunakan format lain.
Misalnya, pada kakas Tensorflow disediakan sebuah cara untuk menyimpan model dalam sebuah file lewat \mintinline{python}{save_model()}.
Oleh karena itu, konfigurasi format model harus dapat menerima jenis model yang berbeda-beda.
Kode~\ref{listing:5} dan Kode~\ref{listing:6} merupakan merupakan beberapa contoh konfigurasi sederhana untuk model dengan format yang berbeda:

\begin{code}
	\yamlcode{resources/files/configurations/model_basic.yml}
	\caption{Contoh spesifikasi model dengan ONNX}\label{listing:5}
\end{code}
\begin{code}
	\yamlcode{resources/files/configurations/model_keras.yml}
	\caption{Contoh spesifikasi model dengan Keras}\label{listing:6}
\end{code}

Dalam rancangan konfigurasi ini, keduanya memiliki parameter yang serupa.
Model yang telah disimpan pada suatu lokasi akan diberikan letak/\textit{path}-nya dalam parameter ``path''.
Penentuan format dalam parameter ``format'' akan menentukan bagaimana kode yang dibangun akan membaca model yang diberikan dengan cara yang sesuai.

\subsection{Konfigurasi Alur Pemrosesan}\label{section:03-processing-pipeline}

Secara umum, terkadang perlu tahapan-tahapan tertentu dari masukan yang diberikan terhadap sistem.
Misalnya, dalam data tabular kadang dilakukan proses \textit{scaling} terhadap data atau melakukan reduksi dimensi dengan menggunakan teknik seperti PCA dan LDA.\@
Masukan berupa citra kadang diperlukan untuk melakukan tahap \textit{image processing} untuk memastikan citra yang dimasukkan cocok untuk sistem.
Kakas ini akan menerima konfigurasi untuk melakukan hal-hal tersebut lewat modul-modul yang didefinisikan dalam konfigurasi.
Kode~\ref{listing:7} merupakan rancangan konfigurasi untuk alur pemrosesan masukan sistem.

\begin{code}
	\yamlcode{resources/files/configurations/pipeline_basic.yml}
	\caption{Contoh spesifikasi pemrosesan data}\label{listing:7}
\end{code}

Sebagai penjelasan singkat, dalam konfigurasi ini terdapat masukan untuk jenis pemrosesan yang akan dilakukan terhadap suatu format tabular.
Selain itu, terdapat konfigurasi untuk memilih data masukan mana yang ingin diproses disertai dengan tujuan.
Informasi lainnya adalah untuk menjadi parameter bagi modul pemrosesan data.
Parameter tersebut akan beragam bergantung pada modul pemrosesan data yang digunakan.

\subsection{Konfigurasi \textit{Interface}}\label{section:03-interface-config}
Sistem secara umum akan menggunakan salah satu atau kedua dari \textit{interface}, yaitu REST atau gRPC.\@
Dengan mempertimbangkan kedua hal tersebut, kakas ini akan berfokus pada kedua \textit{interface} tersebut.
Kode dari sistem yang dibangkitkan tentunya dapat dimodifikasi sesuai kebutuhan dari sistem bila diperlukan perubahan terhadap \textit{interface}.
Kode~\ref{listing:8} merupakan contoh konfigurasi \textit{interface} sederhana.

\begin{code}
	\yamlcode{resources/files/configurations/interface_basic.yml}
	\caption{Contoh spesifikasi sistem dengan \textit{interface} REST}\label{listing:8}
\end{code}

Konfigurasi ini tidak menutup kemungkinan untuk memiliki dua \textit{interface} sekaligus.
Kode~\ref{listing:9} merupakan contoh konfigurasi sistem dengan dua \textit{interface} sekaligus: REST dan gRPC.\@

\begin{code}
	\yamlcode{resources/files/configurations/interface_multiple.yml}
	\caption{Contoh spesifikasi sistem dengan beberapa \textit{interface}}\label{listing:9}
\end{code}
