\section{Studi Kasus Konversi Eksperimen}

Untuk merancang sebuah kakas, penulis telah melakukan konversi eksperimen dari awal hingga menjadi sebuah kode sistem.
Eksperimen pemelajaran mesin dilakukan sedemikian sehingga penulis mendapatkan hal-hal yang dibutuhkan untuk membangun sebuah sistem pemelajaran mesin.
Dalam proses konversi ini, penulis menggunakan bahasa Python dengan kakas-kakas pemelajaran mesin dan pembuatan API.

\subsection{Eksperimen Data Tabular}
Eksperimen ini dilakukan dengan menggunakan kakas scikit-learn.
Untuk API, penulis menggunakan kakas FastAPI dengan tujuan akhir membuat sistem dengan arsitektur REST sederhana.
API hanya terdiri dari satu endpoint saja, yaitu untuk melakukan prediksi terhadap suatu data masukan.

Penulis mengambil dari salah satu contoh eksperimen pemelajaran mesin yang paling sederhana, yaitu melakukan prediksi terhadap data dari penumpang Titanic; apakah penumpang tersebut akan selamat atau tidak dari data yang diketahui dari kecelakaan kapal tersebut.
Penulis melakukan eksperimen seperti biasa, dengan fokus eksperimen adalah dengan memperhatikan bagaimana data diproses dan bagaimana model hasil eksperimen disimpan.
Setelah melakukan eksperimen, penulis mengambil beberapa hal dan kata kunci yang perlu diperhatikan dalam merancang kakas yaitu sebagai berikut:

\begin{enumerate}
	\item Masukan model bergantung pada masukan data pada sistem 
	
	Salah satu yang menjadi poin penting dalam melakukan konversi adalah mengetahui bahwa fitur dari masukan model dan masukan untuk sistem dapat berbeda.
	Sangat mungkin masukan sebuah model lebih banyak daripada masukan pada sistem dan mungkin juga tidak perlu dilakukan perubahan signifikan terhadap masukan kepada sistem.
	Selain itu, pada akhirnya data masukan secara umum perlu dilakukan pemetaan karena kedua masukan ini menentukan masukan pada model.
	Terkait hal ini lebih rincinya dibahas dalam poin-poin berikutnya.

	\item Perlu dilakukan pemrosesan terhadap data masukan
	
	Beberapa tahapan pemrosesan data terkadang diperlukan setelah data masukan diberikan.
	Misalnya, dalam ekesperimen ini terdapat salah satu fitur yang dilakukan \textit{scaling} terhadap data.
	Model pemelajaran mesin memerlukan data yang sudah diterapkan \textit{scaling}, namun kemungkinan besar agar API bisa digunakan secara intuitif, masukan seharusnya berupa nilai aslinya sebelum dilakukan \textit{scaling}.
	Pemrosesan data ini bergantung pada data yang digunakan pada proses pelatihan, sehingga tentunya pada setiap eksperimen proses tersebut akan berbeda-beda.
	
	\item Jumlah masukan dapat berbeda dengan masukan model
	
	Berkaitan dengan poin sebelumnya, masukan untuk API mungkin berbeda dengan masukan model.
	Hal ini terjadi secara umum karena pemrosesan data yang dilakukan dalam sistem.
	Terdapat metode pemrosesan data seperti \textit{one-hot encoding} yang membuat dimensi dari fitur bertambah untuk model.
	Dari sudut pandang API hal tersebut kurang baik untuk diterapkan karena akan mempersulit masukan terhadap sistem.
\end{enumerate}

Berdasarkan studi kasus sederhana ini, penulis membuat format konfigurasi kakas untuk studi kasus ini.
Berikut adalah format konfigurasinya:

\yamlcode{resources/files/configurations/titanic_full.yml}
