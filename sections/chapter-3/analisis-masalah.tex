\section{Analisis Masalah}

Saat ini, konversi dari eksperimen ke sistem yang \textit{production-ready} masih dilakukan secara manual.
Dari ekperimen tersebut, biasanya terdapat proses-proses untuk melakukan pemrosesan data.
Pemrosesan data bisa dilakukan secara langsung sebelum dimasukkan ke dalam model, maupun dilakukan prekomputasi untuk mempercepat proses inferensi dengan mengabaikan faktor masukan data.
Dalam percobaan-percobaan yang memanfaatkan teknologi seperti Kubeflow, dapat digunakan sistem pipeline yang tersedia untuk melakukan eksperimentasi.

Dengan penggunaan pipeline dalam eksperimen pembelajaran mesin, proses eksperimen bisa dilakukan dengan lebih cepat.
Hal tersebut karena pipeline didesain agar bisa digunakan kembali di beberapa proses yang serupa.
Proses yang dimaksud bisa berupa proses untuk melakukan preproses pada data, atau bisa untuk melakukan proses pelatihan model.
Pembuatannya yang membutuhkan script Python secara umum membuat proses pada pipeline ini bersifat atomik serta fleksibel untuk digunakan dalam pelatihan model yang berbeda.

Namun, penggunaan pipeline ini biasanya hanya digunakan untuk eksperimen saja.
Pipeline yang dibuat untuk eksperimen ini biasanya tidak digunakan lagi dalam proses pembuatan sistem inferensi, karena hasil eksperimen biasanya akan disimpan dalam sebuah file yang hanya perlu dibungkus lewat suatu metode komunikasi, seperti yang telah dibahas pada bagian sebelumnya.
Sedangkan, dengan penggunaan pipeline ini sebenarnya mungkin saja dapat membantu agar proses pembuatan sistem lebih mulus.

Dalam sebuah eksperimen yang sukses, model biasanya hanya hanya perlu untuk dibungkus dalam sebuah metode komunikasi sesuai dengan kebutuhan arsitektur, misalnya dengan gRPC atau REST.
Perlu adanya perjanjian antar sistem untuk memastikan objek yang dikirim dan diterima sesuai dengan pihak yang membutuhkan sistem.
Misalnya, seperti dalam gRPC yang menggunakan Protobuf terdapat konvensi perjanjian prosedur dan struktur data yang terlibat, atau JSON Schema dan Swagger dalam REST.
Hal ini juga perlu menjadi perhatian dalam pembuatan sistem pembelajaran mesin.
