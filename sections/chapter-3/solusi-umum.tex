\section{Solusi Umum}

Seperti yang disinggung sebelumnya, konversi eksperimen masih relatif manual dan belum memanfaatkan pipeline yang ada.
Sehingga, terdapat potensi pengembangan untuk membuat suatu kakas yang bisa mengubah eksperimen menjadi sistem yang siap digunakan.
Berdasarkan eksperimen yang dilakukan lewat pipeline, secara teori memungkinkan untuk mengubah komponen-komponen yang ada dalam pipeline tersebut menjadi beberapa bagian dalam suatu sistem.

Dalam sebuah eksperimen yang sukses, model biasanya hanya hanya perlu untuk dibungkus dalam sebuah metode komunikasi sesuai dengan kebutuhan arsitektur, misalnya dengan gRPC atau REST.
Perlu adanya perjanjian antar sistem untuk memastikan objek yang dikirim dan diterima sesuai dengan pihak yang membutuhkan sistem.
Misalnya, seperti dalam gRPC yang menggunakan Protobuf terdapat konvensi perjanjian prosedur dan struktur data yang terlibat, atau JSON Schema dan Swagger dalam REST.
Solusi yang dibuat dapat memperhatikan hal ini sehingga struktur data dan prosedur yang ada bisa dijamin, mengingat juga ada kemungkinan 

Terkait juga dengan masukan data, secara umum data diberikan secara mentah tanpa pemrosesan apapun, tetapi terkadang terdapat data yang berbeda sehinga perlu dilakukan pemrosesan yang berbeda.
Solusi ini dapat memanfaatkan pipeline yang sudah ada dalam membuat suatu alur pemrosesan data di dalam sistem.
Alur pemrosesan didefinisikan secara terpisah agar dapat didefinisikan implementasi yang berbeda-beda tergantung kebutuhan.
Bila diimplementasikan dengan benar, untuk beberapa hal mungkin juga digunakan bahasa yang lebih cepat, sehingga hanya proses inferensi model saja dan beberapa hal saja yang sulit untuk dipindahkan ke bahasa lain.

Sebagai contoh, dalam pemrosesan data tabular umumnya terdapat beberapa tahapan dalam melakukan pemrosesan terhadap data.
Tahapan \textit{preprocessing} yang dimaksud misalnya bisa berupa normalisasi terhadap fitur, pembentukan fitur baru, seleksi fitur, dan transformasi fitur.
Seperti pada transformasi fitur menggunakan metode seperti \textit{Principal Component Analysis} (PCA) atau \textit{Linear Discriminant Analysis} (LDA), tentunya akan memerlukan proses konversi dari data mentah menjadi data dengan dimensi yang berbeda.

