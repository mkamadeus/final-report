\section{Alur Konversi Spesifikasi}

Dalam bagian ini, penulis membahas bagaimana alur konversi dari spesifikasi hingga menjadi suatu kode sistem yang siap digunakan.
Alur konversi ini akan dijadikan landasan dalam proses implementasi kakas.
Gambar~\ref{fig:context} merupakan gambaran alur penggunaan kakas secara umum melalui diagram konteks  dari kakas ini.

\begin{figure}[H]
    \centering
    \includegraphics[width=0.4\textwidth]{resources/images/chapter-3/03-diagram-konteks.drawio.png}
    \caption{Diagram Konteks Kakas}\label{fig:context}
\end{figure}

Untuk lebih rincinya, penulis membuat diagram aliran data (\textit{Data Flow Diagram}) untuk tahap konversi dari spesifikasi.
Hal yang perlu menjadi perhatian utama adalah tahapan konversi kakas dalam melakukan pemetaan data masukan dan \textit{pipeline} pemrosesan data.
Gambar~\ref{fig:dfd} menunjukkan alur penggunaan tiap komponen dalam spesifikasi ketika dilakukan konversi menggunakan kakas. 

\begin{figure}[H]
    \centering
    \includegraphics[width=0.9\textwidth]{resources/images/chapter-3/03-diagram-pembacaan-spesifikasi.drawio.png}
    \caption{\textit{Data Flow Diagram} Kakas}\label{fig:dfd}
\end{figure}