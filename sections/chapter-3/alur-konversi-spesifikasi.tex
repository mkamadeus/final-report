\section{Alur Konversi Spesifikasi}

Bagian ini akan menjelaskan bagaimana pemrosesan terhadap spesifikasi \textit{markup} dilakukan, seperti pada alur yang digambarkan dalam Gambar \ref{fig:03-tool}.
Penjelasan tentang alur konversi mencakup bagaimana suatu bagian dalam spesifikasi berkorespondensi dengan hasil akhir dari kode sistem yang dibangkitkan.
Selain itu, penjelasan ini juga akan mencakup bagaimana urutan pembangkitan spesifikasi dilakukan.

\subsection{Konversi Spesifikasi Masukan}

Dalam bagian \ref{section:03-input-format}, dijelaskan bahwa jenis masukan bisa beragam.
Oleh karena itu, proses konversi dari spesifikasi masukan akan bergantung kepada format masukan yang diperlukan.
Sebagai contoh, untuk data tabular, akan terdapat masukan berupa kolom-kolom yang perlu diubah menjadi bentuk masukan yang sesuai dengan spesifikasi model.
Lain halnya untuk data citra, secara umum hanya diperlukan satu gambar yang perlu diberikan ke dalam sistem untuk diproses, atau banyak gambar yang diproses secara terpisah tanpa terkait satu sama lain.

Dalam rancangan yang dibuat penulis, spesifikasi masukan dalam data tabular akan digunakan sebagai format masukan secara langsung.
Bagian yang perlu dilakukan pemrosesan tersebut akan diubah ketika data diterima oleh sistem.
Pada data citra, masukan akan berupa sebuah gambar yang diunggah ke sistem.
Pemrosesan lanjut pada suatu bagian dibahas pada bagian berikutnya dalam spesifikasi alur pemrosesan. 

\subsection{Konversi Spesifikasi Keluaran}


\subsection{Konversi Spesifikasi Model}

\subsection{Konversi Spesifikasi Alur Pemrosesan}

\subsection{Konversi Spesifikasi \textit{Interface}}
