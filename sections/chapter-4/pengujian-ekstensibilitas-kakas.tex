% chktex-file 26
% chktex-file 21

\section{Pengujian Ekstensibilitas Kakas}

Bagian ini akan membahas pengujian terhadap ekstensibilitas kakas.
Pembahasan bagian ini akan terkait dengan bagaimana implementasi modul dalam kakas dilakukan.
Bila seandaikan diperlukan fitur tambahan untuk kakas, secara teori dan desain dari implementasi kakas ini dimungkinkan untuk ditambahkan fungsionalitas dan modul baru.
Pengujian bertujuan untuk melihat apakah rancangan implementasi yang dibuat berhasil diimplementasikan sehingga kakas dapat ditambahkan fungsionalitasnya, khususnya dalam hal penambahan modul pemrosesan data.

Pengujian ini dilakukan melalui salah satu studi kasus yang telah dilakukan sebelumnya, yaitu kasus yang dibahas pada Bagian~\ref{section:churn-rate-case}.
Terdapat salah satu modul yang belum diimplementasikan pada kakas tersebut, yaitu modul yang dapat melakukan \textit{label encoding}.
Oleh karena itu, penulis menambahkan modul tersebut pada kakas.

Langkah pertama yang dilakukan penulis adalah melihat bagaimana \textit{label encoding} dilakukan di dalam eksperimen.
Dalam eksperimen tersebut, penulis menggunakan kakas \monospace{scikit-learn} yang umum digunakan untuk memproses data.
Terdapat sebuah kelas yang mendefinisikan \textit{label encoding} dalam \textit{library} tersebut.
Penulis mengambil kelas hasil data latih dan mengunduhnya untuk digunakan dalam kakas menggunakan metode yang disediakan oleh \textit{library} tersebut.

Selanjutnya, penulis mencari tahu bagaimana kode untuk menggunakan 

