\section{Pengujian Kakas Konversi}

Bagian ini akan membahas pengujian apa saja yang telah dilakukan penulis.
Pengujian yang dilakukan akan membahas kasus-kasus yang diimplementasikan dalam kasus, yaitu data tabular dan data citra.
Pengujian akan dilakukan terhadap eksperimen dan konfigurasi yang berbeda.
Bagian ini juga akan membahas konfigurasi kakas masing-masing eksperimene yang akan dibuat untuk diberikan kepada kakas.

\subsection{Kasus Data Titanic}

Dataset ini merupakan permasalahan klasik yang digunakan untuk mencoba algoritma pemelajaran mesin pada data tabular.
Informasi mengenai penumpang dari kapal Titanic tertera dalam dataset ini.
Penulis melakukan eksperimen secara pribadi menggunakan kakas \mintinline{bash}{scikit-learn} untuk menguji kakas pada eksperimen pemelajaran mesin berbasis \textit{shallow learning}.
Model hasil eksperimen yang dilakukan akan dikonversikan dengan format ONNX.\@

Dalam eksperimen ini, dilakukan pemrosesan terhadap dataset yang diberikan.
Rincian lebih lengkapnya bisa dilihat pada lampiran.
Berikut adalah beberapa poin penting pemrosesan data yang dilakukan dalam eksprimen ini.

\begin{enumerate}
	\item Melakukan \textit{feature scaling} dengan melakukan standardisasi terhadap fitur umur.
	\item Melakukan \textit{feature selection} dengan membuang kolom-kolom yang tidak begitu relevan terhadap hasil prediksi menurut pandangan penulis dengan tujuan mengurangi dimensi juga.
	\item Melakukan \textit{one-hot encoding} terhadap beberap fitur seperti jenis kelamin dan tempat keberangkatan penumpang.
\end{enumerate}

Berikut ini adalah konfigurasi yang dirancang oleh penulis.

\begin{code}
	\yamlcode{resources/files/configurations/titanic_full.yml}
	\caption{Konfigurasi sistem eksperimen Titanic}
	\label{listing:18}
\end{code}

Selain konfigurasi, file model dan file lainnya yang diperlukan untuk melakukan pemrosesan data juga diperlukan.
Dalam eksperimen ini, terdapat tiga file utama berdasarkan eksperimen yang dilakukan, yaitu file konfigurasi, file model, dan file \textit{scaler} yang digunakan untuk melakukan scaling terhadap umur.
Hasil \textit{scaler} pada tahapaan pelatihan seharusnya digunakan kembali untuk menjamin scaling dilakukan dengan tepat berdasarkan data pelatihan.

Kakas dapat dijalankan dengan menggunakan perintah \mintinline{bash}{./myx -o ./ spec.yaml}, yang artinya memberikan hasil pada direktori ini berdasarkan konfigurasi yang terdapat dalam file \mintinline{bash}{spec.yaml}.
Berikut adalah kode yang dihasilkan oleh kakas.
Selain dari kode, dihasilkan juga file-file lainnya seperti yang disinggung sebelumnya.

\begin{code}
	\pycode{resources/files/code/titanic_result.py}
	\caption{Hasil kode sistem oleh kakas untuk eksperimen Titanic}
	\label{listing:19}
\end{code}

Menggunakan salah satu data uji secara acak, diberikan masukan kepada sistem yang dijalankan.
Berikut ini masukan yang diberikan serta keluaran yang diberikan.

\begin{code}
	\jsoncode{resources/files/tests/titanic_input.json}
	\caption{Masukan sistem eksperimen Titanic}
	\label{listing:20}
\end{code}

\begin{code}
	\jsoncode{resources/files/tests/titanic_output.json}
	\caption{Keluaran sistem eksperimen Titanic}
	\label{listing:21}
\end{code}

\subsection{Kasus Data Churn Rate Karyawan}

Dataset ini adalah dataset yang dipilih penulis yang juga merupakan dataset tabular.
Dataset ini memiliki data karyawan dengan indikator apakah karyawan tersebut akan keluar atau tidak (\textit{churn rate}).
Permasalahan ini serupa dengan permasalahan pada dataset Titanic yang merupakan permasalahan klasifikasi.

Dalam pengujian ini, dilakukan pemrosesan data terhadap beberapa fitur.
Berikut ini adalah daftar pemrosesan terhadap fitur yang dilakukan.

\begin{enumerate}
	\item Melakukan \textit{feature scaling} terhadap beberapa fitur: skor kredit, umur, saldo, jumlah kepemilikan bangunan, jumlah produk, dan perkiraan gaji.
	\item Melakukan \textit{one-hot encoding} terhadap satu fitur, yaitu lokasi.
	\item Melakukan \textit{label encoding} terhadap satu fitur, yaitu jenis kelamin.
\end{enumerate}

Berikut ini adalah konfigurasi yang dirancang oleh penulis.

\begin{code}
	\yamlcode{resources/files/configurations/churn_full.yml}
	\caption{Konfigurasi sistem eksperimen Churn Rate}\label{listing:22}
\end{code}

Seperti sebelumnya, kakas dijalankan dengan cara yang sama.
Direktori mengandung file-file yang diperlukan untuk melakukan \textit{encoding} dan \textit{scaling}.
Kode yang dihasilkan akan menghasilkan kode sistem yang dapat dijalankan.
Berikut ini adalah masukan dan keluaran untuk sistem ini.

\subsection{Kasus Data MNIST}

penjelasan
Konfigurasi
hasil

\subsection{Kasus Data Citra Anjing dan Kucing}

penjelasan
Konfigurasi
hasil

\subsection{Catatan Tambahan Pengujian}

hal yang diuji menyeluruh
temuan lainnya
