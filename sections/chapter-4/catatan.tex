\section{Catatan Tambahan Pengujian}

Eksperimen yang dilakukan oleh penulis dan yang diambil penulis semuanya merupakan eksperimen pemelajaran mesin yang sukses melakukan prediksi.
Tidak semua eksperimen demikian, namun penulis hanya mengambil eksperimen yang berhasil saja untuk pengujian kasus ini.
Kakas ini tidak berhubungan secara langsung dengan proses pelatihan dan pengembangan model pemelajaran mesin.
Seandainya model tidak melakukan prediksi dengan benar, hal tersebut tidak menjadi tanggung jawab dari kakas ini.

Uji kasus yang diambil penulis terbatas dalam permasalahan klasifikasi saja.
Namun, tidak menutup kemungkinan kakas ini dapat digunakan dalam permasalahan lainnya seperti regresi atau \textit{clustering}.
Bila diperlukan, kode untuk kakas dapat ditambah fungsionalitasnya untuk mengatasi kasus-kasus yang belum teratasi.

Kakas ini juga tidak mengatasi bagaimana hasil prediksi dikeluarkan.
Pengguna diberi kebebasan untuk memproses lebih lanjut hasil prediksi dari sistem, karena tiap keluaran model memiliki makna yang berbeda-beda tergantung dari arsitektur suatu model.
Misalnya, dalam permasalahan MNIST yang dibahas sebelumnya keluaran berupa \textit{array} dengan panjang 10 yang merepresentasikan angka dari nol hingga sembilan.